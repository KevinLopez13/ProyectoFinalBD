\documentclass[12pt,letterpaper]{article}
\usepackage[utf8]{inputenc}
\usepackage[spanish,mexico]{babel}
\usepackage[top=2.5cm,bottom=2cm,left=1.8cm,right=2cm]{geometry}
\usepackage{graphicx}
\usepackage{caption}
\usepackage{amssymb}
\usepackage{floatrow}
\usepackage{fancyvrb}
\usepackage{xcolor}
\usepackage{xurl}
\usepackage{pgfgantt}

\PassOptionsToPackage{hyphens}{url}\usepackage{hyperref}
\hypersetup{
	colorlinks=true,
	urlcolor=blue,
	linkcolor=black
}

\pagestyle{headings}

% -- Estructura para insertar imagenes --

%\begin{figure}[H]
%	\centering
%	\includegraphics[width=0.7\linewidth]{img/nombreImg}
%	\caption{Descripción de la imagen.}
%	\label{fig:alias}
%\end{figure}

% --

\begin{document}
	\newgeometry{top=2cm,bottom=2.5cm,left=1cm,right=1cm}
	\begin{titlepage}
		\centering
		\begin{minipage}{0.14\linewidth}
			\includegraphics[width=\linewidth]{img/shieldUnam}
		\end{minipage}
		\begin{minipage}{0.7\linewidth}
			\centering
			{\bfseries\large UNIVERSIDAD NACIONAL AUTÓNOMA DE MÉXICO \par}
			%\vspace{1cm}
			\vfill
			{\scshape\Large Facultad de Ingeniería \par}
			\vfill
			%\vspace{0.7cm}
			{\scshape\Large Ingeniería en Computación \par}
		\end{minipage}
		\begin{minipage}{0.14\linewidth}
			\includegraphics[width=\linewidth]{img/shieldFi}
		\end{minipage}
		
		\centering
		\vspace{1.5cm}
		{\scshape\Large Bases de Datos \par}
		{\scshape\Large grupo: 01\par}
		\vspace{3cm}
		{\scshape\Huge Proyecto final \par}
		\vspace{0.8cm}
		%{\scshape\Huge fff \par}
		\vfill
		{\Large Alumnos: \par}
		\begin{center}
			\begin{tabular}{l}
				$\bullet$ \\
				\\
				$\bullet$ \\
				\\
				$\bullet$ {\Large López González Kevin } \\
				\\
				$\bullet$ \\
				\\
				$\bullet$ \\
			\end{tabular}
		\end{center}
		\vfill
		{\Large Profesor: \par}
		{\Large ING. Fernando Arreola Franco \par}
		\vfill
		{\Large \today \par}
	\end{titlepage}

	\restoregeometry
	\tableofcontents
	\section{Introducción}
	
	\section{Plan de trabajo}
		\subsection{Descripción}
		
		\subsection{Plan de actividades}
		%%\begin{center}
		%%	\begin{tabular}
				
		%%	\end{tabular}
		%%\end{center}
		
		\subsection{Cronograma}
			\begin{ganttchart}[vgrid, x unit=7mm, time slot format=isodate,
				canvas/.style={draw=none},
				title/.append style={fill=blue!10, rounded corners=1mm},
				group/.append style={draw=black, fill=magenta!60},
				bar/.append style={fill=magenta!20}]
				{2021-11-22}{2021-12-11}
				\gantttitlecalendar{year, month=name, day} \\
				
				\ganttgroup{Diseño}{2021-11-22}{2021-11-24}\\
				\ganttbar{M. Conceptual}{2021-11-22}{2021-11-22}\\
				\ganttbar{M. Lógico}{2021-11-22}{2021-11-22} \\
				\ganttbar{M. Físico}{2021-11-23}{2021-11-23} \\
				\ganttbar{Revisión}{2021-11-24}{2021-11-24} \\
				
				\ganttgroup{Implementación}{2021-11-25}{2021-12-03}\\
				\ganttbar{Base de datos}{2021-11-25}{2021-11-28}\\
				\ganttbar{Revisión}{2021-11-29}{2021-11-29}\\
				\ganttbar{Pruebas}{2021-11-30}{2021-12-03}\\
				
				\ganttgroup{Presentación}{2021-11-25}{2021-12-03}\\
				\ganttbar{Página Web}{2021-11-25}{2021-11-28}\\
				\ganttbar{Revisión}{2021-11-29}{2021-11-29}\\
				\ganttbar{Pruebas}{2021-11-30}{2021-12-03}\\
				
				\ganttgroup{Acoplamiento}{2021-12-04}{2021-12-09}\\
				\ganttbar{Desarrollo}{2021-12-04}{2021-12-06}\\
				\ganttbar{Pruebas}{2021-12-07}{2021-12-09}\\
				
				\ganttbar{Documentación}{2021-11-22}{2021-12-09}
			\end{ganttchart}

				
		\subsection{Aportaciones}
	
	\section{Diseño}
		\subsection{Análisis de requerimientos}
		
		\subsection{Modelo conceptual}
			\textbf{Entidades}\par 
			\begin{itemize}
				\item PROVEEDOR: \{\}
				\item CLIENTE: \{\}
				\item PRODUCTO: \{\}
				\item VENTA : \{\}
			\end{itemize}
		
			\textbf{Relaciones}\par
			\begin{itemize}
				\item Un ..
				\item 
			\end{itemize}
			\subsubsection{Modelo Entidad-Relación}
			
		\subsection{Modelo lógico}
			\subsubsection{Representación Intermedia}
			\begin{itemize}
				\item PROVEEDOR: \{ id\_proveedor smallint (PK), nombre varchar 50, razón social varchar 50, estado varchar 50, colonia varchar 50, numero smallint, cp smallint, calle varchar 50\}
				\item TELEFONO: \{teléfono bigint(PK), id\_proveedor smallint (FK)\}
				
				\item INVENTARIO: \{id\_Inventario smallint (PK), precio\_compra decimal (10,2), stock smallint, fecha\_compra date \}
				
				\item SURTE: \{[id\_Provedor smaillint (FK), id\_Inventario smallint (FK)] (PK)\}
				
				\item PRODUCTO: \{cod\_barras integer PK, id\_categoria smallint FK, precio smallint NOT NULL, marca varchar(20) NOT NULL, descripcion varchar(50), id\_inventario smallint (FK)\}
				
				\item CATEGORÍA: \{ id\_categoria smallint PK, tipo varchar(20) NOT NULL\}
				
				\item CLIENTE: \{RFC varchar(13) (PK), nombre varchar(20), ap\_paterno varchar (20), ap\_materno varchar (20) (N), cp smallint, numero smallint, estado varchar (32), calle varchar (32), colonia varchar (32)\}
				
				\item EMAIL: \{RFC varchar(13) (FK), email varchar (64) (PK)\}
				
				\item VENTA: \{id\_venta int(PK), fecha\_venta date, pago\_final decimal(7,2), RFC varchar(13)(FK)\}
				
				\item CONTIENE: \{ [cod\_barras int , id\_venta int](PK)(FK), precioTotalArt decimal(7,2), cantidad articulo int\} 
			\end{itemize}
			
			\subsubsection{Transformación de MER a MR}
			
			\subsubsection{Modelo Relacional}
			
			\subsubsection{Normalización}
			
		\subsection{Modelo físico}
		
			\subsubsection{IaaS}
		
	\section{Implementación}
	
		\subsection{Códigos}
		
		\subsection{DDL}
	
	\section{Presentación}
		\subsection{Página Web}
	
	\section{Conclusiones}
		\begin{itemize}
			\item 
				\subitem 
				
			\item López González Kevin
				\subitem Bla bla bla

			\item 
				\subitem
		\end{itemize}
	
\end{document}



















