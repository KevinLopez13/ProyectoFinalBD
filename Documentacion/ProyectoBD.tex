\documentclass[12pt,letterpaper]{article}
\usepackage[utf8]{inputenc}
\usepackage[spanish,mexico]{babel}
\usepackage[top=2.5cm,bottom=2cm,left=1.8cm,right=2cm]{geometry}
\usepackage{graphicx}
\usepackage{caption}
\usepackage{amssymb}
\usepackage{floatrow}
\usepackage{fancyvrb}
\usepackage{xcolor}
\usepackage{xurl}
\usepackage{pgfgantt}

\PassOptionsToPackage{hyphens}{url}\usepackage{hyperref}
\hypersetup{
	colorlinks=true,
	urlcolor=blue,
	linkcolor=black
}

\pagestyle{headings}

\begin{document}
	\newgeometry{top=2cm,bottom=2.5cm,left=1cm,right=1cm}
	\begin{titlepage}
		\centering
		\begin{minipage}{0.14\linewidth}
			\includegraphics[width=\linewidth]{img/shieldUnam}
		\end{minipage}
		\begin{minipage}{0.7\linewidth}
			\centering
			{\bfseries\large UNIVERSIDAD NACIONAL AUTÓNOMA DE MÉXICO \par}
			%\vspace{1cm}
			\vfill
			{\scshape\Large Facultad de Ingeniería \par}
			\vfill
			%\vspace{0.7cm}
			{\scshape\Large Ingeniería en Computación \par}
		\end{minipage}
		\begin{minipage}{0.14\linewidth}
			\includegraphics[width=\linewidth]{img/shieldFi}
		\end{minipage}
		
		\centering
		\vspace{1.5cm}
		{\scshape\Large Bases de Datos \par}
		{\scshape\Large grupo: 01\par}
		\vspace{3cm}
		{\scshape\Huge Proyecto final \par}
		\vspace{0.8cm}
		%{\scshape\Huge fff \par}
		\vfill
		{\Large Alumnos: \par}
		\begin{center}
			\begin{tabular}{l}
				$\bullet$ \\
				\\
				$\bullet$ {\Large López Aniceto Saúl Isaac }\\
				\\
				$\bullet$ {\Large López González Kevin } \\
				\\
				$\bullet$ \\
				\\
				$\bullet$ {\Large Ponce Soriano Armando }\\
			\end{tabular}
		\end{center}
		\vfill
		{\Large Profesor: \par}
		{\Large ING. Fernando Arreola Franco \par}
		\vfill
		{\Large \today \par}
	\end{titlepage}

	\restoregeometry
	\tableofcontents

	\newpage
	\section{Introducción}
	
	\section{Plan de trabajo}
		\subsection{Descripción}
		
		\subsection{Plan de actividades}

		\newpage
		\subsection{Cronograma}
			\begin{ganttchart}[vgrid, x unit=7mm, time slot format=isodate,
				canvas/.style={draw=none},
				title/.append style={fill=blue!10, rounded corners=1mm},
				group/.append style={draw=black, fill=magenta!60},
				bar/.append style={fill=magenta!20}]
				{2021-11-21}{2021-12-09}
				\gantttitlecalendar{year, month=name, day} \\
				
				\ganttgroup{Diseño}{2021-11-21}{2021-11-22}\\
				\ganttbar{Modelo Conceptual}{2021-11-21}{2021-11-21}\\
				\ganttbar{Modelo Lógico}{2021-11-21}{2021-11-21} \\
				\ganttbar{Revisión}{2021-11-22}{2021-11-22} \\
				
				\ganttgroup{Implementación}{2021-11-23}{2021-12-01}\\
				\ganttbar{Modelo Físico}{2021-11-23}{2021-11-25}\\
				\ganttbar{Revisión}{2021-11-26}{2021-11-26}\\
				\ganttbar{Corrección de errores}{2021-11-27}{2021-11-29}\\
				\ganttbar{Pruebas}{2021-11-30}{2021-12-01}\\
				
				\ganttgroup{Presentación}{2021-11-23}{2021-12-01}\\
				\ganttbar{Página Web}{2021-11-23}{2021-11-25}\\
				\ganttbar{Revisión}{2021-11-26}{2021-11-26}\\
				\ganttbar{Corrección de errores}{2021-11-27}{2021-11-29}\\
				\ganttbar{Pruebas}{2021-11-30}{2021-12-01}\\
				
				\ganttgroup{Acoplamiento}{2021-12-02}{2021-12-08}\\
				\ganttbar{Desarrollo}{2021-12-02}{2021-12-04}\\
				\ganttbar{Pruebas}{2021-12-05}{2021-12-05}\\
				\ganttbar{Revisión}{2021-12-06}{2021-12-06}\\
				\ganttbar{Corrección de errores}{2021-12-07}{2021-12-08}\\

				\ganttbar{Documentación}{2021-11-22}{2021-12-08}
			\end{ganttchart}

				
		\subsection{Aportaciones}
		\begin{center}
			\begin{tabular}{c|c|c|c|c|c}
				& Diseño & Implementación & Presentación & Acoplamiento & Documentación\\ \hline
				Kevin López & $\checkmark$ & & $\checkmark$ & $\checkmark$ & $\checkmark$ \\
			\end{tabular}
		\end{center}
	
	\section{Diseño}
		\subsection{Análisis de requerimientos}
		
		\subsection{Modelo conceptual}
			\textbf{Entidades}\par 
			\begin{itemize}
				\item PROVEEDOR: \{ \underline{id\_Proveedor}, razón social, domicilio (estado, código postal, colonia, calle y número), nombre, teléfonos \}
				\item CLIENTE: \{\underline{RFC}, nomre (nombre, ap\_Paterno, ap\_Materno), domicilio (estado, código postal, colonia, calle y número), emails \}
				\item INVENTARIO: \{\underline{id\_Inventario}, precio\_compra, fecha\_compra, cantidad\_ejemplares \}
				
				\item PRODUCTO: \{\underline{código\_Barras}, marca, descripción, precio, categoria\}
				\item VENTA : \{\underline{num\_venta}, fecha\_venta, pago\_Total, cantidad\_articulo, pago\_total\_Articulo \}
			\end{itemize}
		
			\textbf{Relaciones}\par
			\begin{itemize}
				\item Un proveedor surte a muchos inventarios.
				\item Un inventario es surtido por muchos proveedores. \\
				\item Un inventario almacena muchos productos.
				\item Un producto es almacenado por un inventario.\\
				\item Una venta contiene muchos productos.
				\item Un producto es contenido es muchas ventas.\\
				\item Un cliente concreta muchas ventas.
				\item Una venta es concretada por un cliente.
				
			\end{itemize}
			\subsubsection{Modelo Entidad-Relación}
			\begin{figure}[H]
				\centering
				\includegraphics[width=\linewidth]{img/MER}
				\caption{Modelo Entidad-Relación.}
			\end{figure}
			
			
		\subsection{Modelo lógico}
			\subsubsection{Representación Intermedia}
			\begin{itemize}
				\item PROVEEDOR: \{ id\_proveedor smallint (PK), nombre varchar 50, razón social varchar 50, estado varchar 50, colonia varchar 50, numero smallint, cp smallint, calle varchar 50\}
				\item TELEFONO: \{teléfono bigint(PK), id\_proveedor smallint (FK)\}
				
				\item INVENTARIO: \{id\_Inventario smallint (PK), precio\_compra decimal (10,2), stock smallint, fecha\_compra date \}
				
				\item SURTE: \{[id\_Provedor smaillint (FK), id\_Inventario smallint (FK)] (PK)\}
				
				\item PRODUCTO: \{cod\_barras integer PK, id\_categoria smallint FK, precio smallint NOT NULL, marca varchar(20) NOT NULL, descripcion varchar(50), id\_inventario smallint (FK)\}
				
				\item CATEGORÍA: \{ id\_categoria smallint PK, tipo varchar(20) NOT NULL\}
				
				\item CLIENTE: \{RFC varchar(13) (PK), nombre varchar(20), ap\_paterno varchar (20), ap\_materno varchar (20) (N), cp smallint, numero smallint, estado varchar (32), calle varchar (32), colonia varchar (32)\}
				
				\item EMAIL: \{RFC varchar(13) (FK), email varchar (64) (PK)\}
				
				\item VENTA: \{id\_venta int(PK), fecha\_venta date, pago\_final decimal(7,2), RFC varchar(13)(FK)\}
				
				\item CONTIENE: \{ [cod\_barras int , id\_venta int](PK)(FK), precioTotalArt decimal(7,2), cantidad articulo int\} 
			\end{itemize}
			
			\subsubsection{Transformación de MER a MR}
			
			\subsubsection{Modelo Relacional}
			
			\subsubsection{Normalización}
		
	\section{Implementación}
		\subsection{Modelo físico}
		
			\subsubsection{IaaS}
	
		\subsection{Códigos}
		
		\subsection{DDL}
	
	\section{Presentación}
		Una interfaz gráfica es un programa que nos permite manipular información a través de objetos gráficos que proporcionen un entorno visual, con el fin facilitar la interacción del usuario con la computadora.\par
		En este caso, se optó por desarrollar una página web como interfaz gráfica que permita la manipulación de información de nuestra base de datos.
		
		\subsection{Django}
		Django es un framework de Python de alto nivel que permite diseñar aplicaciones web de una forma rápida, limpia y pragmática. Además, ayuda a los desarrolladores a evitar muchos errores de seguridad comunes, como la inyección de SQL, las secuencias de comandos entre sitios, la falsificación de solicitudes entre sitios y el secuestro de clics.
			
			\subsubsection{Mapeo Relacional de Objetos}
			El Mapeo Relacional de Objetos o ORM (\textit{Object Relational Mapping}), es una tecnología que soluciona el desajuste entre las bases de datos relacionales y orientadas a objetos.\par 
			Por lo general, asigna una clase a una tabla uno a uno. Cada instancia de la clase corresponde a un registro en la tabla y cada atributo de la clase corresponde a cada campo en la tabla. ORM proporciona una asignación a la base de datos, en lugar de escribir código SQL directamente, solo es necesario manipular los datos de la base de datos como un objeto operativo.\par 
			Si bien, el ORM es una herrmienta muy util, no se utilizará en este proyecto, ya que preferimos escribir directamente la sentencia SQL para comunicarnos con la base de datos.
			
			\subsubsection{Ejecutar SQL personalizado directamente}
			El objeto \textbf{django.db.connection} representa la conexión por defecto entre django y la base de datos. Las funciones que utilizamos para la comunicación con la base de datos son las siguientes:
			\begin{itemize}
				\item \textbf{connection.cursor()}
					\subitem Para obtener un objeto cursor.
					
				\item \textbf{cursor.execute(sql, [params])}
					\subitem Para ejecutar sentencias SQL.
					
				\item \textbf{cursor.fetchall()}
					\subitem Para devolver las filas resultantes de la consulta.
			\end{itemize}
		
		\subsection{Diseño}
		
	
	\section{Conclusiones}
		\begin{itemize}
			\item 
				\subitem 
				
			\item López González Kevin
				\subitem Bla bla bla

			\item 
				\subitem
		\end{itemize}

	
\end{document}



















